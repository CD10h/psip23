\documentclass{VUMIFPSkursinis}
\usepackage{float}
\usepackage{hyperref}
\usepackage{algorithmicx}
\usepackage{algorithm}
\usepackage{algpseudocode}
\usepackage{amsfonts}
\usepackage{amsmath}
\usepackage{bm}
\usepackage{caption}
\usepackage{color}
\usepackage{graphicx}
\usepackage{listings}
\usepackage{subcaption}
\usepackage{wrapfig}
\usepackage{biblatex}
\usepackage{microtype}
\usepackage{Procesai}


% Titulinio aprašas
\university{Vilniaus universitetas}
\faculty{Matematikos ir informatikos fakultetas}
\institute{Informatikos institutas}  % Užkomentavus šią eilutę - institutas neįtraukiamas į titulinį
\department{Programų sistemų studijų programa}
\papertype{Kursinis darbas}
\title{Programų sistemų kūrimo metodų tyrimas}
\titleineng{Investigation of Methods of Software Development}
\status{4 kurso ... grupės studentas}
\author{Vardenis Pavardenis}
% \secondauthor{Vardonis Pavardonis}   % Pridėti antrą autorių
\supervisor{prof. habil. dr. Vardaitis Pavardaitis}
% \addsignatureplaces{} % prideda parašų vietas tituliniame puslapyje
\date{Vilnius – \the\year}

% \bibliography{bibliografija}


\begin{document}
\maketitle

% \tableofcontents

\sectionnonum{Įvadas}
Įvade apibūdinamas darbo tikslas, temos aktualumas ir siekiami rezultatai.
Darbo įvadas neturi būti dėstymo santrauka. Įvado apimtis 1–-2 puslapiai.

\section{Procesai}

\begin{procesas}
	\pavadinimas{KOD}{ProcessName}
	\tikslas{Paskaiciuoti skaiciaus n faktoriala}
	\vykdytojai{
		\item Vykdytojas1
	}
	\veiklos{
		\item Veikla1
		\item Veikla2
	}

	\ieitys{
		\ieitis{KOD}{Ieitis1}
		\ieitis{KOD2}{Ieitis1}
	}
	\isvestys{
		\isvestis{KOD}{ Isvestis1}
		\isvestis{KOD}{ Isvestis2}
	}
\end{procesas}


% Other: gal ir nereik?

% - Verifikacija ( su klientu?)
% - Bendra perziura
% - Auditavimas
% - Problemu sprendimas
% - Apmokymo procesas
% - HR procesai
% - Prototipavimas?
% - Prog. Įrangos reikalavimų analizė



\begin{procesas}

	\pavadinimas{PR\_PREPOFF}{Projekto pasiūlymo procesas}

	\tikslas{Paruošti projekto pasiūlymą, kuris bus įteiktas klientui}

	\ieitys{
		\ieitis{KlPI}{ Kliento projekto idėja}
		\ieitis{SPD}{ Senų projektų duomenys }}
	\isvestys{
		\isvestis{PrPas}{ Projekto pasiūlymas }}
	\veiklos{
		\item Įvykdoma preliminari analizė, kurios metu nustatoma:
		\item Preliminari kaina
		\item Preliminari apimtis
		\item Preliminarus laikas, kuris reikalingas projektui įgyvendinti
		\item Paruošiamas pasiūlymas klientui }
	\vykdytojai{
		\item a }

\end{procesas}

\begin{procesas}

	\pavadinimas{PR\_OFFER}{Projekto pasiūlymo pateikimas klientui}

	\tikslas{pateikti klientui pasiūlymą ir susitarti jį vykdyti}

	\veiklos{
		\item Pasirašom sutartį }
	\ieitys{
		\ieitis{PR\_PAS}{ Projekto pasiūlymas klientui }}
	\isvestys{
		\isvestis{KONTR} {Kontraktas }
	}

\end{procesas}


\begin{procesas}
	\pavadinimas{PR\_INIC}{Projekto inicijavimas}

	\tikslas{Pasiruošti projekto planavimo fazei pradėti gautą projektą ar projekto etapą. Išanalizuoti poreikius. neresuotąsias šalis, jų įtaką ir poveikį projektui.}
	\ieitys{
		\ieitis{KOD}{ placeholder}
	}
	\isvestys{
		\isvestis{PASK}{ Projekto paskyra (PR--INIC1)}
		\isvestis{PROJPLAN}{ Projekto planas (PR--INIC2)}
	}
\end{procesas}




\begin{procesas}

	\pavadinimas{PR\_PLAN}{ Projekto planavimas}

	\tikslas{	 Paruošti planą sėkmingam projekto valdymui ir produkto sukūrimui.  }
	\vykdytojai{
		\item Projekto vadovas
		\item Programinės įrangos inžinierius
	}
	\veiklos{
		\item Paruošti projekto planą
		\item Išanalizuoti projekto apimtį (veiklas, reikalingas sėkmingam projekto vykdymui), paruošti apimties valdymo planą su paskirtais resursais.
		\item Paruošti/atnaujinti reikalavimų valdymo planą
		\item Paruošti suinteresuotųjų šalių integravimo į projekto procesą planą
		\item Patikrinti ir patvirtinti projekto planąn
		\item Kokybės audito ataskaita
	}
	\isvestys{
		\isvestis{PLANDOC}{ Projekto plano dokumentas}
		\isvestis{PRP}{ Kokybės audito ataskaita}
		\isvestis{TESTPLAN}{ Testavimo planas}
	}
\end{procesas}

\begin{procesas}

	\pavadinimas{PR\_ANA}{Projekto reikalavimų analizė}

	\tikslas{Išanalizuoti ir specifikuoti projekto reikalavimus}

	\ieitys{
		\ieitis{PLANDOC}{ Projekto plano dokumentas}
		\ieitis{CLIREQ}{ Kliento reikalavimai }}
	\isvestys{
		\isvestis{CLIREQ}{ Kliento reikalavimai}
	}
	\veiklos{
		\item Kartu su užsakovu išsiaiškinti aukšto lygmens sistemos reikalavimus
		\item Sukurti aukšto lygmens reikalavimų dokumentą
	}
	\vykdytojai{
		\item Projekto vadovas
		\item Užsakovas }

\end{procesas}



\begin{procesas}

	\pavadinimas{PR\_TECHREQ}{Techniniai reikalavimai}

	\tikslas{Transformuoti kliento reikalavimus į techninius sitemos reikalavimus}

	\ieitys{
		\ieitis{CLIREQ}{ Kliento reikalavimai}
		\ieitis{asdasd} {Projekto apribojimai ( laikas etc) }
	}
	\isvestys{
		\isvestis{TESTPLAN}{ Testavimo planas}
		\isvestis{PROGDOC}{ Projekto eigos ataskaita}
		\isvestis{ARCHSPEC}{ Sistemos architektūros specifikacija }
		\isvestis{TECHSPEC}{ Techninė specifikacija }
	}
	\vykdytojai{

		\item Projekto vadovas
		\item Programinės įrangos inžinierius
		\item Programuotojai
		\item DevOps (?)
	}
	\veiklos{
		\item Suformuoti aukšo lygmens sistemos architektūrą
		\item Atmesti reikalavimus, kurių neįmanoma įgyvendinti per laiką ir apimtį
	}

\end{procesas}


\begin{procesas}

	\pavadinimas{PR\_PROJ}{Projektavimas (PROJ)}

	\tikslas{Paruošti sistemos architektūrą bei techninio įgyvendinimo planą.}


	\veiklos{
		\item Paruošti sistemos architektūrą
		\item Paruošti techninę specifikaciją
		\item Paruošti testavimo planą
		\item Paruošti projekto eigos ataskaitą
		\item Paruošti projektinę aplinką
	}
	\ieitys{
		\ieitis{KOD}{ Projekto paskyra (PR--INIC1)}
		\ieitis{KOD}{ Projekto plano dokumentas (PR--PLAN1) }}
	\isvestys{

		\isvestis{KOD}{ Projekto vykdymo resursų reikalavimai (PR--AN5)}
	}
\end{procesas}





\begin{procesas}

	\pavadinimas{PR\_REV}{Bendra peržiūra}

\end{procesas}

\begin{procesas}

	\pavadinimas{PR\_ARCH}{Architektūros dizainas}

\end{procesas}


\begin{procesas}

	\pavadinimas{PR\_DEPLAN}{Projekto įvedimo į eksploataciją planavimas }
	\tikslas{Gamybinės aplinkos parengimas produkto eksploatacijai, diegimo plano parengimas }
	\vykdytojai{
		\item Dev Ops (Infrastruktūros inžinieriai?)+
		\item Projekto vadovas }
	\veiklos{
		\item Gamybinės aplinkos parengimas produkto eksploatacijai
		\item Infrastruktūros parengimas debesijos infrastruktūroje
		\item Diegimo plano sudarymas
		\item Pagal poreikį
		\item Bandomosios eksploatacijos plano ir įgyvendinimo rengimas
		\item Duomenų migravimas
		\item Sukonfigūravimas viešai prieigai }
	\ieitys{
		\ieitis{KOD}{ Sistemos architektūros specifikacija}
		\ieitis{KOD}{ Techninė specifikacija}
		\ieitis{KOD}{ Projekto dokumentacija }}
	\isvestys{
		\isvestis{KOD}{Projekto įvedimo į eksploataciją planas}
	}
\end{procesas}


\begin{procesas}

	\pavadinimas{KOD}{Pasiruosimas kitai iteracijai}

	\tikslas{Paskirstyti užduotis programuotojams}

	\ieitys{
		\ieitis{KOD}{ a }}
	\isvestys{
		\isvestis{KOD}{ a }}
	\veiklos{
		\item a }
	\vykdytojai{
		\item a }

\end{procesas}

\begin{procesas}

	\pavadinimas{KOD}{Uzduoties programavimas}

\end{procesas}

\begin{procesas}

	\pavadinimas{PR\_TTEST}{ Užduoties testavimas ir kokybės užtikrinimas }
	\tikslas{	Įsitikinti, ar programų sistema atitinka reikalavimus }
	\vykdytojai{
		\item Testuotojai
		\item Projekto vadovas
		\item Programuotojai }
	\veiklos{
		\item Testavimo plano vykdymas
		\item Testavimo rezultatų parengimas }
	\ieitys{
		\ieitis{SRC}{Programinis kodas}
		\ieitis{TASK}{Užduotys}
		\ieitis{TPLAN}{Testavimo planas}
		\ieitis{TECHSPEC}{Reikalavimų techninė specifikacija}
	}
	\isvestys{
		\isvestis{BUGLST}{Defektų ataskaita}
		\isvestis{TREZ}{Testavimo rezultatų ataskaita}
		\isvestis{TPLAN}{Testavimo planas}
		\isvestis{TASK}{Užduotys}
	}
\end{procesas}

\begin{procesas}

	\pavadinimas{KOD}{Uzduoties release}

\end{procesas}

\begin{procesas}

	\pavadinimas{KOD}{ Refinement}

	% - Gryniname tech. reikalavimus iš user stories
	% - Pr
\end{procesas}

\begin{procesas}

	\pavadinimas{KOD}{ Sprint retro}

	% - Patikrinam, ar telpam į laiką
	% - Perkeliam užduotis, kurių nespėjom
	% - Aptariam?
\end{procesas}


\begin{procesas}

	\pavadinimas{KOD}{ Continuous integration}

\end{procesas}



\begin{procesas}

	\pavadinimas{PR\_DEPL}{ Sistemos įdiegimas }
	\tikslas{	Programų sistemos perkėlimas į produkcinę aplinką ir parengimas eksploatacijai }
	\vykdytojai{
		\item Programuotojai
		\item Infrastruktūros inžinieriai
		\item Produkto palaikymo komanda }
	\veiklos{
		\item Perkelti parengtą sistemą į produkcinę aplinką
		\item Parengti resursus, reikalingus eksploatuoti programų sistemą }
	\ieitys{
		\ieitis{PRODREL}{ Programinė įranga}
		\ieitis{KOD}{ Projekto įvedimo į eksploataciją planas}
		\ieitis{KOD}{ NF Reikalavimai }}
	\isvestys{
		\isvestis{KOD}{ Paruošta eksploatacijai programų sistema?}
		\isvestis{KOD}{ Projekto eigos ataskaita }}
\end{procesas}



\begin{procesas}

	\pavadinimas{PR\_VER}{Verifikacijos procesas}
	\tikslas{Įsitikinti, kad sistema atitinka funkcinius ir nefunkcinius reikalavimus bei kokybės standartus. Peržiūros procesas yra svarbus siekiant užtikrinti, kad projekto rezultatai atitiktų pradinius tikslus ir lūkesčius.}
	\vykdytojai{
		\item Projekto vadovas
		\item Programuotojai
		\item Testuotojai
		\item Projekto rėmėjas/savininkas (sponsor)
	}
	\veiklos{
		\item Reikalavimų įvykdymo peržiūra
		\item Įrangos techninio dizaino peržiūra
		\item Dokumentacijos peržiūra }
	\ieitys{
		\ieitis{REQSPEC}{ Reikalavimų specifikacija }
		\ieitis{ARCHSPEC}{ Sistemos architektūros specifikacija }
		\ieitis{TECHSPEC}{ Techninė specifikacija }
		\ieitis{TPLAN}{ Testavimo planas }
	}
	\isvestys{
		\isvestis{KOD}{ Kokybės audito ataskaita }
		\isvestis{KOD}{ Priėmimo dokumentas }
	}

\end{procesas}

\begin{procesas}

	\pavadinimas{PR\_SHIPIT}{ Perleidimas klientui}

\end{procesas}


\begin{procesas}

	\pavadinimas{PR\_HELP}{Produkto garantinio aptarnavimo procesas}
	\tikslas{Visiems paslaugos rezultato elementams (sudėtinėms dalims, pagal Lietuvos Respublikos Civilinį kodeksą, toliau –- Civilinis kodeksas), kuriems pagal Civilinį kodeksą būtų galimybė sutartiniu įsipareigojimu suteikti kokybės garantijos terminą, pirkimo sutartimi suteikiamas ne mažesnis nei 12 mėnesių kokybės garantijos terminas, kurio pradžia laikoma konkrečios paslaugos dalies priėmimo-perdavimo akto pasirašymo diena. Garantinės priežiūros paslaugos apima sukurtos ir modernizuotos programinės įrangos sutrikimų šalinimą bei Perkančiosios organizacijos atsakingų asmenų konsultavimą.}
	\vykdytojai{
		\item Projekto vadovas
		\item Programuotojai
		\item Užsakovas }
	\veiklos{
		\item Įvertinti, ar kliento pranešta klaida bus taisoma ( pvz not a bug - a feature)
		\item Jeigu, taip, sukurti užduotį su reikalavimais
		% gal split i kita task apacioj
		\item Užsakovo konsultavimas produkto veikimo, naudojimo bei tobulinimo klausimais
		\item Incidentų identifikavimas, klasifikavimas, registravimas priežiūros tarnybos (angl. Help Desk) programinėję įrangoje ir šalinimas
		\item Atsiradusių produkto trikdžių sprendimas
		\item Kokybės garantija taikoma:
		\item Visiems paslaugos teikimo metu sukurtiems ir modifikuotiems funkcionalumams pagal suderintą projektinę dokumentaciją
		\item Trečiųjų šalių komponentų, naudojamų produkte, atnaujinimo darbams
		\item Bet kokiems įsilaužimams į programinės įrangos ir / ar atskirų jos elementų sandarą, kurie galėtų daryti įtaką elementų veikimui }
	\ieitys{
		\ieitis{pro}{ Projekto dokumentacija (PR--PLAN1)}
		\ieitis{KOD}{ Projekto plano dokumentas (PR--PLAN1) }}
	\isvestys{
		\isvestis{KOD}{  Projekto garantinio aptarnavimo planas}
		\isvestis{KOD}{ Žemiau pateikiamas principinis incidento apdorojimo procesas }}
\end{procesas}

\begin{procesas}

	\pavadinimas{PR\_END}{ Užbaigimas}
	\tikslas{Užbaigti sistemos priežiūrą}
	\ieitys{
		\ieitis{KOD}{ a }} %????
	\isvestys{
		\isvestis{KOD}{ Pranešimas apie nebepalaikymą}
		\isvestis{KOD}{ Formalus doc }}
	\veiklos{
		\item Nustojama stebėti sistema
		\item Klientui pranešama, kad garantinis aptarnavimas baigėsi }
	\vykdytojai{
		\item a }

\end{procesas}



\begin{procesas}

	\pavadinimas{RESAQ}{ Resursų įgijimas }

	\tikslas{Įgyti ir parengti resursus, skirtą pasiekti kitų procesų tikslus}

	\vykdytojai{
		\item Skyriaus darbuotojai
		\item Komandų vadovai
		\item Projekto vadovas
		\item Finansų skyriaus atstovai/aquisitions
	}
	\veiklos{
		\item Jeigu nuspręsta pirkti resursus, pateikti finansams reikalingų resursų sąrašą
		\item Jeigu nuspręsta resursus pagaminti kompanijos viduje, pateikti pasiūlymą resurso gamybai (t.y. einam i reikalavimų analizę, programuojam, testuojam, releasinam)
	}
	\ieitys{
		\ieitis{KOD}{ Reikalingų resursų sąrašas}
		\ieitis{KOD}{ Resursų reikalavimų ataskaita}
	}
	\isvestys{
		\isvestis{RPD}{Resursai}}

\end{procesas}

\begin{procesas}

	\pavadinimas{KOD}{ Proceso peržiūra}

\end{procesas}

\begin{procesas}

	\pavadinimas{PR\_DOC}{Dokumentacijos procesas}
	%  branda naudoja dok. standartus 
	\tikslas{Dokumentuoti}

	\ieitys{
		\ieitis{SYS}{ Programų sistema}
		\ieitis{KLIREQ}{ Klientų pageidavimai}
		\ieitis{TECHREQ}{ Techniniai reikalavimai}
		\ieitis{CONFIG}{Konfigūracija}
		\ieitis{TASK}{ Užduotys }}
	\isvestys{
		\isvestis{DOCS}{Dokumentacija}}
	\veiklos{
		\item Sukurti dokumentus \ldots
		\item Patikrinti dokumentus
		\item Prižiūrėti dokumentus
	}
	\vykdytojai{
		\item Support team
		\item Programming team
		\item Management team }

\end{procesas}




\forlistloop{\printkodlist}{\prodkodlist}

\subsection{Poskyris}


\end{document}
