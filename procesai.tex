\begin{procesas}
	\pavadinimas{KOD}{ProcessName}
	\tikslas{Paskaiciuoti skaiciaus n faktoriala}
	\vykdytojai{
		\item Vykdytojas1
	}
	\veiklos{
		\item Veikla1
		\item Veikla2
	}

	\ieitys{
		\ieitis{KOD}{Ieitis1}
		\ieitis{KOD2}{Ieitis1}
	}
	\isvestys{
		\isvestis{KOD}{ Isvestis1}
		\isvestis{KOD}{ Isvestis2}
	}
\end{procesas}

\begin{procesas}

	\pavadinimas{KOD}{Projekto pasiūlymo procesas}

	\tikslas{Paruošti projekto pasiūlymą, kuris bus įteiktas klientui}

	\ieitys{
		\ieitis{KOD}{ Kliento projekto idėja}
		\ieitis{KOD}{ Senų projektų duomenys }}
	\isvestys{
		\isvestis{KOD}{ Projekto pasiūlymas }}
	\veiklos{
		\item Įvykdoma preliminari analizė, kurios metu nustatoma:
		\item Preliminari kaina
		\item Preliminari apimtis
		\item Preliminarus laikas, kuris reikalingas projektui įgyvendinti
		\item Paruošiamas pasiūlymas klientui }
	\vykdytojai{
		\item a }

\end{procesas}

\begin{procesas}

	\pavadinimas{KOD}{Projekto pasiūlymo pateikimas klientui}

	\tikslas{pateikti klientui pasiūlymą ir susitarti jį vykdyti}

	\veiklos{
		\item Pasirašom sutartį }
	\ieitys{
		\ieitis{KOD}{ Projekto pasiūlymas klientui }}
	\isvestys{
		\item Kontraktas }

\end{procesas}

\begin{procesas}

	\pavadinimas{KOD}{Planavimas}

\end{procesas}

\begin{procesas}

	\pavadinimas{KOD}{Projekto reikalavimai}

	\tikslas{x}

	\ieitys{
		\ieitis{KOD}{ Projekto vykdymo planas}
		\ieitis{KOD}{ Kliento reikalavimai }}
	\isvestys{
		\isvestis{KOD}{ Kliento reikalavimai (? ar geriau ataskaita ?)}
	}
	\veiklos{
		\item Išsiaiškinti kliento norus (idealius)
		\item Sukurti aukšto lygmens reikalavimų dokumentą }
	\vykdytojai{
		\item a }

\end{procesas}

\begin{procesas}

	\pavadinimas{KOD}{Techniniai reikalavimai}

	\tikslas{Transformuoti kliento reikalavimus į techninius sitemos reikalavimus}

	\ieitys{
		\ieitis{KOD}{ Kliento reikalavimai (? ar geriau ataskaita ?)}
	}
	\isvestys{
		\isvestis{KOD}{ Sistemos architektūra}
		\isvestis{KOD}{ Sistemos techniniai reikalavnice errorschninius reikalavimus }}
	\vykdytojai{
		\item a }

\end{procesas}

\begin{procesas}

	\pavadinimas{KOD}{Bendra peržiūra}

\end{procesas}

\begin{procesas}

	\pavadinimas{KOD}{Architektūros dizainas}

\end{procesas}

% Iter: %

\begin{procesas}

	\pavadinimas{KOD}{Pasiruosimas kitai iteracijai}

	\tikslas{Paskirstyti užduotis programuotojams}

	\ieitys{
		\ieitis{KOD}{ a }}
	\isvestys{
		\isvestis{KOD}{ a }}
	\veiklos{
		\item a }
	\vykdytojai{
		\item a }

\end{procesas}

\begin{procesas}

	\pavadinimas{KOD}{Uzduoties programavimas}

\end{procesas}

\begin{procesas}

	\pavadinimas{KOD}{Uzduoties testavimas}

\end{procesas}

\begin{procesas}

	\pavadinimas{KOD}{Uzduoties release}

\end{procesas}

\begin{procesas}

	\pavadinimas{KOD}{ Refinement}

	% - Gryniname tech. reikalavimus iš user stories
	% - Pr
\end{procesas}

\begin{procesas}

	\pavadinimas{KOD}{ Sprint retro}

	% - Patikrinam, ar telpam į laiką
	% - Perkeliam užduotis, kurių nespėjom
	% - Aptariam?
\end{procesas}

% Pos-iter:
\begin{procesas}

	\pavadinimas{KOD}{ Bendra peržiūra}

\end{procesas}

\begin{procesas}

	\pavadinimas{KOD}{ Deployment}

\end{procesas}

\begin{procesas}

	\pavadinimas{KOD}{ Perleidimas klientui}

\end{procesas}

\begin{procesas}

	\pavadinimas{KOD}{ Garantinis aptarnavimas}

	\ieitys{
		\ieitis{KOD}{ Kliento klaidos pranešimas}
		\ieitis{KOD}{ Reikalavimai }}
	\isvestys{
		\isvestis{KOD}{ Užduoti}
		\isvestis{KOD}{ Reikalavimai }}
	\veiklos{
		\item Įvertinti, ar kliento pranešta klaida bus taisoma
		\item Jeigu, taip, sukurti užduotį su reikalavimais }
	\vykdytojai{
		\item a }

\end{procesas}

\begin{procesas}

	\pavadinimas{KOD}{ Užbaigimas}

	\ieitys{
		\ieitis{KOD}{ a }}
	\isvestys{
		\isvestis{KOD}{ Pranešimas apie nebepalaikymą}
		\isvestis{KOD}{ Formalus doc }}
	\veiklos{
		\item Nustojama stebėti sistema
		\item Klientui pranešama, kad garantinis aptarnavimas baigėsi }
	\vykdytojai{
		\item a }

\end{procesas}

% Helpers:

\begin{procesas}

	\pavadinimas{KOD}{ Continuous integration}

\end{procesas}

\begin{procesas}

	\pavadinimas{KOD}{ Resursų įgijimas}

\end{procesas}

\begin{procesas}

	\pavadinimas{KOD}{ Proceso peržiūra}

\end{procesas}

\begin{procesas}

	\pavadinimas{KOD}{ Config'ai}

\end{procesas}

\begin{procesas}

	\pavadinimas{KOD}{ Iššūkių sprendimas}

\end{procesas}

\begin{procesas}

	\pavadinimas{KOD}{ Dokumentacija}
	%  branda naudoja dok. standartus 
	\tikslas{Dokumentuoti}

	\ieitys{
		\ieitis{KOD}{ Programų sistema}
		\ieitis{KOD}{ Klientų pageidavimai}
		\ieitis{KOD}{ Reikalavimai}
		\ieitis{KOD}{ Konfigūracija?}
		\ieitis{KOD}{ Užduotys }}
	\isvestys{
		\isvestis{KOD}{ Dokumentacija }}
	\veiklos{
		\item Sukurti dokumentus \ldots
		\item Patikrinti dokumentus
		\item Prižiūrėti dokumentus }
	\vykdytojai{
		\item Support team
		\item Programming team
		\item Management team }

\end{procesas}




% Other: gal ir nereik?

% - Verifikacija ( su klientu?)
% - Bendra perziura
% - Auditavimas
% - Problemu sprendimas
% - Apmokymo procesas
% - HR procesai
% - Prototipavimas?
% - Prog. Įrangos reikalavimų analizė


% SENI PROCESAI IRGI CIA %

\begin{procesas}
	\pavadinimas{KOD}{Projekto inicijavimas (PR--INIC)}

	\tikslas{Pasiruošti projekto planavimo fazei pradėti gautą projektą ar projekto etapą. Išanalizuoti poreikius. neresuotąsias šalis, jų įtaką ir poveikį projektui.}
	\ieitys{
		\ieitis{KOD}{ placeholder}
	}
	\isvestys{
		\isvestis{KOD}{ Projekto paskyra (PR--INIC1)}
		\isvestis{KOD}{ Projekto planas (PR--INIC2)}
	}
\end{procesas}

\begin{procesas}

	\pavadinimas{KOD}{ Projekto analizė (PR--ANA) }

	\tikslas{Išanalizuoti ir specifikuoti projekto reikalavimus}

	\vykdytojai{
		\item Projekto vadovas
		\item Užsakovas }
	\veiklos{
		\item Kartu su užsakovu paruošti reikalavimų specifikaciją
		\item Vykdymo aplinkos ir licencijų reikalavimų paruošimas
	}
	\ieitys{

		\ieitis{KOD}{ Projekto paskyra (PR--INIC1)}
	}
	\isvestys{

		\isvestis{KOD}{ Reikalavimų specifikacija (PR--AN1)}
	}
\end{procesas}

\begin{procesas}

	\pavadinimas{KOD}{Projektavimas (PROJ)}

	\tikslas{Paruošti sistemos architektūrą bei techninio įgyvendinimo planą.}

	\vykdytojai{

		\item Projekto vadovas
		\item Programinės įrangos inžinierius
		\item Programuotojai
		\item DevOps (?)
	}
	\veiklos{
		\item Paruošti sistemos architektūrą
		\item Paruošti techninę specifikaciją
		\item Paruošti testavimo planą
		\item Paruošti projekto eigos ataskaitą
		\item Paruošti projektinę aplinką
	}
	\ieitys{
		\ieitis{KOD}{ Projekto paskyra (PR--INIC1)}
		\ieitis{KOD}{ Projekto plano dokumentas (PR--PLAN1) }}
	\isvestys{
		% \isvestis{KOD}{ Testavimo planas (PR--AN&PROJ5)}
		% \isvestis{KOD}{ Projekto eigos ataskaita (PR--AN&PROJ6)}
		\isvestis{KOD}{ Sistemos architektūros specifikacija (PR--AN2)}
		\isvestis{KOD}{ Techninė specifikacija (PR--AN4)}
		\isvestis{KOD}{ Projekto vykdymo resursų reikalavimai (PR--AN5)}
	}
\end{procesas}


\begin{procesas}

	\pavadinimas{KOD}{ Projekto planavimas (PR--PLAN) }

	\tikslas{	 Paruošti planą sėkmingam projekto valdymui ir produkto sukūrimui.  }
	\vykdytojai{
		\item Projekto vadovas
		\item Programinės įrangos inžinierius
	}
	\veiklos{
		\item Paruošti projekto planą
		\item Išanalizuoti projekto apimtį (veiklas, reikalingas sėkmingam projekto vykdymui), paruošti apimties valdymo planą su paskirtais resursais.
		\item Paruošti/atnaujinti reikalavimų valdymo planą
		\item Paruošti suinteresuotųjų šalių integravimo į projekto procesą planą
		\item Patikrinti ir patvirtinti projekto planąn
		\item Kokybės audito ataskaita
	}
	\isvestys{
		\isvestis{KOD}{ Projekto plano dokumentas (PR--PLAN1)}
		\isvestis{KOD}{ Kokybės audito ataskaita (PR--PLAN2)}
		\isvestis{KOD}{ Testavimo planas (PR--PLAN3)}
	}
\end{procesas}

\begin{procesas}

	\pavadinimas{KOD}{Programų sistemos vystymas}

	\tikslas{Programų sistemos dalies vykdymas pagal suplanuotas užduotis}
	\vykdytojai{
		\item Projekto Vadovas
		\item Programuotojai
		\item Testuotojai
		\item Devops }
	\veiklos{
		\item Programų sistemos vystymas pagal specifikacijas
		\item Vystymo progreso sekimas
		\item Automatinių testų paruošimas
		\item Automatinių testų paleidimas
		\item 3.4 metu rastu defektų taisymas}
	\ieitys{
		\ieitis{KOD}{ Reikalavimų specifikacija}
		\ieitis{KOD}{ Sistemos architektūros specifikacija}
		\ieitis{KOD}{ Techninė specifikacija}
		\ieitis{KOD}{ Projekto plano dokumentas}
		\ieitis{KOD}{ Testavimo planas}
		\ieitis{KOD}{ Programinis kodas}
		\ieitis{KOD}{ Defektų sąrašas}
		\ieitis{KOD}{ Automatiniai programų sistemos dalies testai}
		\ieitis{KOD}{ Projekto eigos ataskaita}
	}
	\isvestys{
		\isvestis{KOD}{ Programinis kodas}
		\isvestis{KOD}{ Automatiniai programų sistemos dalies testai}
		\isvestis{KOD}{ Testavimo planas}
		\isvestis{KOD}{ Projekto eigos ataskaita}
	}
\end{procesas}

\begin{procesas}

	\pavadinimas{KOD}{ Programų sistemos testavimas ir kokybės užtikrinimas }
	\tikslas{	Įsitikinti, ar programų sistema atitinka reikalavimus }
	\vykdytojai{
		\item Testuotojai
		\item Projekto vadovas
		\item Programuotojai }
	\veiklos{
		\item Testavimo plano vykdymas
		\item Testavimo rezultatų parengimas }
	\ieitys{
		\ieitis{KOD}{ Programinis kodas}
		\ieitis{KOD}{ Testavimo planas}
		\ieitis{KOD}{ Reikalavimų specifikacija}
		\ieitis{KOD}{ Sistemos architektūros specifikacija }}
	\isvestys{
		\isvestis{KOD}{ Defektų ataskaita}
		\isvestis{KOD}{ Testavimo rezultatų ataskaita}
		\isvestis{KOD}{ Testavimo planas}
		\isvestis{KOD}{ Projekto eigos ataskaita }}
\end{procesas}

\begin{procesas}

	\pavadinimas{KOD}{ (PRE) Projekto įvedimo į eksploataciją planavimas }
	\tikslas{Gamybinės aplinkos parengimas produkto eksploatacijai, diegimo plano parengimas }
	\vykdytojai{
		\item Dev Ops (Infrastruktūros inžinieriai?)+
		\item Projekto vadovas }
	\veiklos{
		\item Gamybinės aplinkos parengimas produkto eksploatacijai
		\item Infrastruktūros parengimas debesijos infrastruktūroje
		\item Diegimo plano sudarymas
		\item Pagal poreikį
		\item Bandomosios eksploatacijos plano ir įgyvendinimo rengimas
		\item Duomenų migravimas
		\item Sukonfigūravimas viešai prieigai }
	\ieitys{
		\ieitis{KOD}{ Sistemos architektūros specifikacija}
		\ieitis{KOD}{ Techninė specifikacija}
		\ieitis{KOD}{ Projekto dokumentacija }}
	\isvestys{
		\isvestis{KOD}{ Projekto įvedimo į eksploataciją planas}
	}
\end{procesas}

\begin{procesas}

	\pavadinimas{KOD}{ Sistemos įdiegimas }
	\tikslas{	Programų sistemos perkėlimas į produkcinę aplinką ir parengimas eksploatacijai }
	\vykdytojai{
		\item Programuotojai
		\item Infrastruktūros inžinieriai
		\item Produkto palaikymo komanda }
	\veiklos{
		\item Perkelti parengtą sistemą į produkcinę aplinką
		\item Parengti resursus, reikalingus eksploatuoti programų sistemą }
	\ieitys{
		\ieitis{KOD}{ Programinė įranga}
		\ieitis{KOD}{ Projekto įvedimo į eksploataciją planas}
		\ieitis{KOD}{ NF Reikalavimai }}
	\isvestys{
		\isvestis{KOD}{ Paruošta eksploatacijai programų sistema?}
		\isvestis{KOD}{ Projekto eigos ataskaita }}
\end{procesas}

\begin{procesas}

	\pavadinimas{KOD}{ Peržiūra }
	\tikslas{Įsitikinti, kad sistema atitinka funkcinius ir nefunkcinius reikalavimus bei kokybės standartus. Peržiūros procesas yra svarbus siekiant užtikrinti, kad projekto rezultatai atitiktų pradinius tikslus ir lūkesčius.}
	\vykdytojai{
		\item Projekto vadovas
		\item Programuotojai
		\item Testuotojai
		\item Projekto rėmėjas/savininkas (sponsor) }
	\veiklos{
		\item Reikalavimų įvykdymo peržiūra
		\item Įrangos techninio dizaino peržiūra
		\item Dokumentacijos peržiūra }
	\ieitys{
		% \ieitis{KOD}{ Reikalavimų specifikacija (PR--AN&PROJ1)}
		% \ieitis{KOD}{ Sistemos architektūros specifikacija (PR--AN&PROJ2)}
		% \ieitis{KOD}{ Techninė specifikacija (PR--AN&PROJ4)}
		% \ieitis{KOD}{ Testavimo planas (PR--AN&PROJ5) }
		\ieitis{KOD}{ placeholder}
	}
	\isvestys{
		\isvestis{KOD}{ Kokybės audito ataskaita}
		\isvestis{KOD}{ Priėmimo dokumentas }}

\end{procesas}

\begin{procesas}

	\pavadinimas{KOD}{ (PGAP) Produkto garantinio aptarnavimo procesas}
	\tikslas{Visiems paslaugos rezultato elementams (sudėtinėms dalims, pagal Lietuvos Respublikos Civilinį kodeksą, toliau – Civilinis kodeksas), kuriems pagal Civilinį kodeksą būtų galimybė sutartiniu įsipareigojimu suteikti kokybės garantijos terminą, pirkimo sutartimi suteikiamas ne mažesnis nei 12 mėnesių kokybės garantijos terminas, kurio pradžia laikoma konkrečios paslaugos dalies priėmimo-perdavimo akto pasirašymo diena. Garantinės priežiūros paslaugos apima sukurtos ir modernizuotos programinės įrangos sutrikimų šalinimą bei Perkančiosios organizacijos atsakingų asmenų konsultavimą.}
	\vykdytojai{
		\item Projekto vadovas
		\item Programuotojai
		\item Užsakovas }
	\veiklos{
		\item Užsakovo konsultavimas produkto veikimo, naudojimo bei tobulinimo klausimais
		\item Incidentų identifikavimas, klasifikavimas, registravimas priežiūros tarnybos (angl. Help Desk) programinėję įrangoje ir šalinimas
		\item Atsiradusių produkto trikdžių sprendimas
		\item Kokybės garantija taikoma:
		\item Visiems paslaugos teikimo metu sukurtiems ir modifikuotiems funkcionalumams pagal suderintą projektinę dokumentaciją
		\item Trečiųjų šalių komponentų, naudojamų produkte, atnaujinimo darbams
		\item Bet kokiems įsilaužimams į programinės įrangos ir / ar atskirų jos elementų sandarą, kurie galėtų daryti įtaką elementų veikimui }
	\ieitys{
		\ieitis{KOD}{ Projekto dokumentacija (PR--PLAN1)}
		\ieitis{KOD}{ Projekto plano dokumentas (PR--PLAN1) }}
	\isvestys{
		\isvestis{KOD}{  Projekto garantinio aptarnavimo planas}
		\isvestis{KOD}{ Žemiau pateikiamas principinis incidento apdorojimo procesas }}
\end{procesas}

\begin{procesas}

	\pavadinimas{KOD}{ (PPP) Produkto priežiūros procesas (PALYGINTI SU ARMANTO PALAIKYMU)}

	\tikslas{Bazinės priežiūros darbai apima serverių infrastruktūroje veikiančių komponentų virtualių mašinų ir juose veikiančios sisteminės programinės įrangos priežiūrą.}

	\vykdytojai{
		\item Dev Ops
		\item Programuotojai }
	\veiklos{
		\item Stebėsenos įrankio įdiegimas, stebėsenos rodiklių sukonfigūravimas, kritinių parametrų ir informavimo sukonfigūravimas
		\item Stebėsenos rodiklių stebėsena (periodiškai)
		\item Virtualių mašinų ir sisteminės programinės įrangos naudojimo trūkumų šalinimas (pagal poreikį)
		\item Duomenų bazių veikimo priežiūra
		\item Duomenų bazių duomenų atsarginių kopijų darymo proceso stebėsena (periodiškai)
		\item Sisteminės programinės įrangos veikimo žurnalų įrašų (angl. Log) priežiūra
		\item Programinės įrangos atnaujinimas ir saugumo spragų šalinimas }
	\ieitys{
		\ieitis{KOD}{ Sistemos architektūros specifikacija}
		\ieitis{KOD}{ Techninė specifikacija}
		\ieitis{KOD}{ Projekto dokumentacija }}
	\isvestys{
		\isvestis{KOD}{ Projekto dokumentacija}
		\isvestis{KOD}{ Periodiškos ataskaitos???}
	}
\end{procesas}

\begin{procesas}

	\pavadinimas{KOD}{ Palaikymas (panašu į garantinį) }

	\tikslas{	Eksploatuojamos programinės įrangos priežiūra, siekiant užtikrinti tinkamą veikimą. }

	\vykdytojai{
		\item Palaikymo komanda
		\item Suinteresuotos šalys
	}
	\veiklos{
		\item Klaidų registravimas
		\item Klaidų prioretizavimas pagal jų poveikį ir skubumą.
		\item Klaidų šalinimas
		\item Sistemos rodiklių stebėjimas
	}
	\ieitys{
		\ieitis{KOD}{ placeholder}
	}
	\isvestys{
		\isvestis{KOD}{ placeholder}
	}
\end{procesas}

\begin{procesas}

	\pavadinimas{KOD}{ Resursų įgijimas }

	\tikslas{Įgyti ir parengti resursus, skirtą pasiekti kitų procesų tikslus}

	\vykdytojai{
		\item Skyriaus darbuotojai
		\item Komandų vadovai
		\item Projekto vadovas
		\item Finansų skyriaus atstovai/aquisitions
	}
	\veiklos{
		\item Jeigu nuspręsta pirkti resursus, pateikti finansams reikalingų resursų sąrašą
		\item Jeigu nuspręsta resursus pagaminti kompanijos viduje, pateikti pasiūlymą resurso gamybai (t.y. einam i reikalavimų analizę, programuojam, testuojam, releasinam)
	}
	\ieitys{
		\ieitis{KOD}{ Reikalingų resursų sąrašas}
		\ieitis{KOD}{ Resursų reikalavimų ataskaita}
	}
	\isvestys{
		\isvestis{KOD}{ Resursų pirkimo dokumentas}
		\isvestis{KOD}{ Reikalingo resurso techn./funkc/arch specifikacija }}
\end{procesas}