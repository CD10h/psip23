
\begin{procesas}

	\pavadinimas{PR\_PREPOFF}{Projekto pasiūlymo procesas}

	\tikslas{Paruošti projekto pasiūlymą, kuris bus įteiktas klientui}

	\ieitys{
		\ieitis{KlPI}{ Kliento projekto idėja}
		\ieitis{SPD}{ Senų projektų duomenys }}
	\isvestys{
		\isvestis{PrPas}{ Projekto pasiūlymas }}
	\veiklos{
		\item Įvykdoma preliminari analizė, kurios metu nustatoma:
		\item Preliminari kaina
		\item Preliminari apimtis
		\item Preliminarus laikas, kuris reikalingas projektui įgyvendinti
		\item Paruošiamas pasiūlymas klientui }
	\vykdytojai{
		\item a }

\end{procesas}

\begin{procesas}

	\pavadinimas{PR\_OFFER}{Projekto pasiūlymo pateikimas klientui}

	\tikslas{pateikti klientui pasiūlymą ir susitarti jį vykdyti}

	\veiklos{
		\item Pasirašom sutartį }
	\ieitys{
		\ieitis{PR\_PAS}{ Projekto pasiūlymas klientui }}
	\isvestys{
		\isvestis{KONTR} {Kontraktas }
	}

\end{procesas}


\begin{procesas}
	\pavadinimas{PR\_INIC}{Projekto inicijavimas}

	\tikslas{Pasiruošti projekto planavimo fazei pradėti gautą projektą ar projekto etapą. Išanalizuoti poreikius. neresuotąsias šalis, jų įtaką ir poveikį projektui.}
	\ieitys{
		\ieitis{KOD}{ placeholder}
	}
	\isvestys{
		\isvestis{PASK}{ Projekto paskyra (PR--INIC1)}
		\isvestis{PROJPLAN}{ Projekto planas (PR--INIC2)}
	}
\end{procesas}




\begin{procesas}

	\pavadinimas{PR\_PLAN}{ Projekto planavimas}

	\tikslas{	 Paruošti planą sėkmingam projekto valdymui ir produkto sukūrimui.  }
	\vykdytojai{
		\item Projekto vadovas
		\item Programinės įrangos inžinierius
	}
	\veiklos{
		\item Paruošti projekto planą
		\item Išanalizuoti projekto apimtį (veiklas, reikalingas sėkmingam projekto vykdymui), paruošti apimties valdymo planą su paskirtais resursais.
		\item Paruošti/atnaujinti reikalavimų valdymo planą
		\item Paruošti suinteresuotųjų šalių integravimo į projekto procesą planą
		\item Patikrinti ir patvirtinti projekto planąn
		\item Kokybės audito ataskaita
	}
	\isvestys{
		\isvestis{PLANDOC}{ Projekto plano dokumentas}
		\isvestis{PRP}{ Kokybės audito ataskaita}
		\isvestis{TESTPLAN}{ Testavimo planas}
	}
\end{procesas}

\begin{procesas}

	\pavadinimas{PR\_ANA}{Projekto reikalavimų analizė}

	\tikslas{Išanalizuoti ir specifikuoti projekto reikalavimus}

	\ieitys{
		\ieitis{PLANDOC}{ Projekto plano dokumentas}
		\ieitis{CLIREQ}{ Kliento reikalavimai }}
	\isvestys{
		\isvestis{CLIREQ}{ Kliento reikalavimai}
	}
	\veiklos{
		\item Kartu su užsakovu išsiaiškinti aukšto lygmens sistemos reikalavimus
		\item Sukurti aukšto lygmens reikalavimų dokumentą
	}
	\vykdytojai{
		\item Projekto vadovas
		\item Užsakovas }

\end{procesas}



\begin{procesas}

	\pavadinimas{PR\_TECHREQ}{Techniniai reikalavimai}

	\tikslas{Transformuoti kliento reikalavimus į techninius sitemos reikalavimus}

	\ieitys{
		\ieitis{CLIREQ}{ Kliento reikalavimai}
		\ieitis{asdasd} {Projekto apribojimai ( laikas etc) }
	}
	\isvestys{
		\isvestis{TESTPLAN}{ Testavimo planas}
		\isvestis{PROGDOC}{ Projekto eigos ataskaita}
		\isvestis{ARCHSPEC}{ Sistemos architektūros specifikacija }
		\isvestis{TECHSPEC}{ Techninė specifikacija }
	}
	\vykdytojai{

		\item Projekto vadovas
		\item Programinės įrangos inžinierius
		\item Programuotojai
		\item DevOps (?)
	}
	\veiklos{
		\item Suformuoti aukšo lygmens sistemos architektūrą
		\item Atmesti reikalavimus, kurių neįmanoma įgyvendinti per laiką ir apimtį
	}

\end{procesas}


\begin{procesas}

	\pavadinimas{PR\_PROJ}{Projektavimas (PROJ)}

	\tikslas{Paruošti sistemos architektūrą bei techninio įgyvendinimo planą.}


	\veiklos{
		\item Paruošti sistemos architektūrą
		\item Paruošti techninę specifikaciją
		\item Paruošti testavimo planą
		\item Paruošti projekto eigos ataskaitą
		\item Paruošti projektinę aplinką
	}
	\ieitys{
		\ieitis{KOD}{ Projekto paskyra (PR--INIC1)}
		\ieitis{KOD}{ Projekto plano dokumentas (PR--PLAN1) }}
	\isvestys{

		\isvestis{KOD}{ Projekto vykdymo resursų reikalavimai (PR--AN5)}
	}
\end{procesas}





\begin{procesas}

	\pavadinimas{PR\_REV}{Bendra peržiūra}

\end{procesas}

\begin{procesas}

	\pavadinimas{PR\_ARCH}{Architektūros dizainas}

\end{procesas}


\begin{procesas}

	\pavadinimas{PR\_DEPLAN}{Projekto įvedimo į eksploataciją planavimas }
	\tikslas{Gamybinės aplinkos parengimas produkto eksploatacijai, diegimo plano parengimas }
	\vykdytojai{
		\item Dev Ops (Infrastruktūros inžinieriai?)+
		\item Projekto vadovas }
	\veiklos{
		\item Gamybinės aplinkos parengimas produkto eksploatacijai
		\item Infrastruktūros parengimas debesijos infrastruktūroje
		\item Diegimo plano sudarymas
		\item Pagal poreikį
		\item Bandomosios eksploatacijos plano ir įgyvendinimo rengimas
		\item Duomenų migravimas
		\item Sukonfigūravimas viešai prieigai }
	\ieitys{
		\ieitis{KOD}{ Sistemos architektūros specifikacija}
		\ieitis{KOD}{ Techninė specifikacija}
		\ieitis{KOD}{ Projekto dokumentacija }}
	\isvestys{
		\isvestis{KOD}{Projekto įvedimo į eksploataciją planas}
	}
\end{procesas}
